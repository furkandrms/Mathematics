\documentclass{article}
\usepackage[utf8]{inputenc}
\usepackage[turkish]{babel}
\usepackage[T1]{fontenc}
\usepackage{amsmath}
\usepackage{amsfonts}
\usepackage{amssymb}

\begin{document}

\( (X, d_X) \) ve \( (Y, d_Y) \) metrik uzaylar ve \( f: X \to Y \) bir fonksiyon olsun. Eğer \( f \) bire bir, örten ve \( X \) üzerinde sürekli ise ve \( Y \) üzerinde sürekli bir ters fonksiyona sahipse (yani \( f^{-1} \), \( Y \) üzerinde sürekli ise) \( f \)'ye bir homeomorfizm ya da topolojik dönüşüm denir.

Eğer \( (X, d_X) \) metrik uzayından \( (Y, d_Y) \) metrik uzayı üzerine en az bir homeomorfizm varsa \( X \) ve \( Y \) homeomorfiktir ya da homeomorf dur denir. Bu durumda \( Y \)'ye \( X \)'in homeomorfik görüntüsü denir. Burada "üzerinde", örten yani \( f(X) = Y \) anlamındadır.

\end{document}
